% Goals for the overall development
\documentclass[11pt]{article}
\usepackage[margin = 0.5 in]{geometry}
\setlength\parindent{0pt}
\title{DeutschLerner Project Goals}
\author{Corbin Foucart}
\date{}

\begin{document}

\maketitle

\section{Interface \& GUI}
Rough Ideas:

\begin{enumerate}
\item 2-3 pages: menu and quiz page
\end{enumerate}

\subsection{Menu}
The idea here is that the user has the ability to access all of the features of the application from this menu.
Perhaps some basic stuff about recently taken quizzes and stats. The menu should be a landing page that
allows the user to move to other pages. Page Structure: \\

\begin{enumerate}
\item Menu
  \begin{enumerate}
    \item Take quiz page
      \begin{enumerate}
      \item Right now let's focus on building something that just tests the user on all of the questions in
        the database. It might be better for applying the memory algorithms.
      \item This would be some type of gui page that simply has a text field for the question and allows 
        the user text boxes to type the answers. For genders, this might include two text boxes. For verbs,
        I can imagine this consisting of a single text field for regular verbs and several text fields for 
        irregular verbs.
      \end{enumerate}
    \item Database editing
  \end{enumerate}
\end{enumerate}

\subsection{Quiz}

\section{Underlying Structure}
The idea is that all data can be appropriately saved to comma separated value files 
and later read in by the program.

\subsection{Question structure}
There should ideally be several types of question: nouns [article + noun + plural], adjectives [word], regular verbs [word], irregular verbs [3 verb forms], verb / preposition combos [infinitive + preposition]. Delimiters for each value are as follows:

\begin{enumerate}
\item Noun - N : ``der Stock, St{\"o}cke'', use ``-'' for when no plural exists
\item Regular Verb - RV : ``rudern''
\item Irregular Verb - IV
\item Adjective - A
\item Preposition - P
\item Preposition Verb combo - VP
\item Phrase - P
\end{enumerate}

This will be achieved by keeping a comma separated value sheet. Each line will represent a single question: \\

[question type delimiter], [question grouping number], [question string], [answer string] \*

The question string will always be displayed. The answer string will be parsed and will quiz the user appropriately.

\section{Probabilistic Question Choosing}

There are two relevant arrays, a questions array and a completed array. At the start of the quiz, all questions are in the questions array. Even past questions with scores above the cuttoff threshold will be asked once. Once a question exceeds some criterion, it moves into the completed array, and can not be asked again during the course of the quiz. \\

Within the question array, each question is assigned a score based on how well the user has answered the question in the past. A question starts at score 0 and gets closer to 1 (infinity?) as the user answers the question correctly. Each score maps to a weighted porability for the question, so that questions with scores closer to 0 are more likely to be asked. \\

The individual unweighted probabilities (don't sum to 1) are calculated:

$$ w_i = 1 - s_i $$

Then we have that:

$$ p_i = \frac{w_i}{\sum_j w_j} $$

Now the individual probabilities sum to 1. We can create an array corresponding to the question array, with the cumulative sum. Each entry corresponds to the `cutoff' for each question. Now a random number is generated between 0 and 1, inclusive. The array is searched for the element of the list containing a number greater than the random number generated. The element preceeding that one corresponds to the correct question to choose. \\

However, it is possible due to round off error that the weighted probability vector can sum to 0.99 rather than 1. Since the random interval is [0,1], it is then possible to choose a probability larger than any possible value. I think cutoff wise this is a non issue, but keep this here in case we need some edge case testing.


\section{Building Goals}
\begin{enumerate}
\item build csv writer
\item build csv reader
\item build csv merger
\item combine the three into a console driven database addition program
\item write a parser that can parse things into the question text fields
\item write a console based program that just displays the stuff without user 
  input.
\item write the quizzing program with the math integration. Should be totally a console program 
  at this point.
\item design and write the gui. Integrate it into the project
\item color the gui to look cool.
\end{enumerate}

%Notes:
%>>> import datetime
%>>> a = datetime.datetime.now()
%>>> b = datetime.datetime.now()
%>>> c = b - a
%datetime.timedelta(0, 8, 562000)
%>>> divmod(c.days * 86400 + c.seconds, 60)
%(0, 8)      # 0 minutes, 8 seconds


\end{document}